%!TEX root =  main.tex

\newcommand{\punt}[1]{}
\newcommand{\calU}{{\cal U}}
\newcommand{\calC}{{\cal C}}
\newcommand{\calB}{{\cal B}}
\newcommand{\calS}{{\cal S}}
\newcommand{\calF}{{\cal F}}
\newcommand{\calD}{{\cal D}}






%\usepackage{ntheorem}

%\makeatletter
%\renewtheoremstyle{plain}% Adds automatic line break, if heading is too long
%  {\item{\theorem@headerfont ##1\ ##2\theorem@separator}{\bf.}~}
%  {\item{\theorem@headerfont ##1\ ##2\ (##3)\theorem@separator}{\bf.}~}
%\makeatother
%
%{\theoremheaderfont{\upshape\bfseries}
% \theorembodyfont{\normalfont\em}
%\newtheorem{definition}{Definition}}
%\newtheorem{observation}{Observation}

\newtheorem{problem}{Problem}
\newtheorem{claim}{Claim}

\makeatletter
\def\@copyrightspace{\relax}
\makeatother


\newcommand{\defn}[1]       {{\textit{\textbf{\boldmath #1}}}\xspace}
% \newcommand{\pparagraph}[1]{\vspace{0.09in}\noindent{\bf \boldmath #1.}} 
% \newcommand{\poly}{\mbox{poly}}
% \newcommand{\polylog}{\mathrm{polylog}}
% \newcommand{\sort}{\mathrm{sort}}
% \newcommand{\scan}{\mathrm{scan}}

\DeclareMathOperator\polylog{polylog}
\DeclareMathOperator\poly{poly}
\DeclareMathOperator\sort{sort}
\DeclareMathOperator\scan{scan}



%%%% VARIABLE NAMES %%%%

\newcommand{\Ns}{N}
\newcommand{\Nk}{u}
%% MAB: do we want stream to be $S$,  $\calS$, or something else? 
\newcommand{\stream}{S}

\newcommand{\graph}{\mathcal{G}}
\newcommand{\nodes}{\mathcal{V}}
\newcommand{\edges}{\mathcal{E}}
\newcommand{\nodesize}{V}
\newcommand{\edgesize}{E}
\newcommand{\graphstream}{S}
\newcommand{\streamlength}{N}
\newcommand{\sketch}{\mathcal{S}}
\newcommand{\blocksize}{B}
\newcommand{\memsize}{M}
\newcommand{\cachesize}{C}
\newcommand{\cachelinesize}{\mathcal{L}}
\newcommand{\disksize}{D}
\newcommand{\streamelement}{s}
\newcommand{\indexsubset}{b}
\newcommand{\support}[1]{supp(#1)}
\newcommand{\prob}[1]{ \Pr \left [ #1 \right ]}
\newcommand{\veclength}{n}
\newcommand{\nodegroup}{\mathcal{U}}
\newcommand{\charvec}{f}
\newcommand{\bin}{bin}
\newcommand{\nodesizebound}{U}
\newcommand{\sketchsize}{\phi}
% \newcommand{\sketchsize}{\log^2(\nodesize)}

\newcommand{\Boruvka}{Bor\r{u}vka\xspace}


\newcommand{\algname}{\textsc{StreamingCC}\xspace}
\newcommand{\sysname}{\textsc{Landscape}\xspace}
\newcommand{\graphzep}{\textsc{GraphZeppelin}\xspace}
\newcommand{\modelname}{hybrid streaming }
\newcommand{\sketchname}{\textsc{CameoSketch}\xspace}
\newcommand{\sketchnames}{\textsc{CameoSketches}\xspace}
\newcommand{\cubesketch}{\textsc{CubeSketch}\xspace}
\newcommand{\cubesketches}{\textsc{CubeSketches}\xspace}
\newcommand{\oldsketchname}{$\ell_0$ sketch\xspace}
\newcommand{\batch}{batch\xspace}
\newcommand{\treename}{pipeline hypertree\xspace}
\newcommand{\Treename}{Pipeline Hypertree\xspace}
\newcommand{\graphworkername}{Graph Worker}
\newcommand{\dsuname}{\textsc{GreedyCC}\xspace}

%\renewcommand{\subparagraph}[1]{\smallskip
%\newcommand{\subparagraph}[1]{\smallskip
%\noindent
%\emph{#1 }}



%%API stuff
\newcommand{\un}{\textunderscore}
\newcommand{\ie}{i.e.,\xspace}
\newcommand{\eg}{e.g.,\xspace}
\newcommand{\initialize}[1]{\textsc{initialize}(#1)\xspace}
\newcommand{\edgeupdate}[1]{\textsc{edge\_update}(#1)\xspace}
\newcommand{\bufferinsert}[1]{\textsc{buffer\_insert}(#1)\xspace}
\newcommand{\dobatchupdate}{\textsc{do\_batch\_update}()\xspace}
\newcommand{\getbatch}{\textsc{get\_batch}()\xspace}
\newcommand{\sketchbatch}[1]{\textsc{update\_sketch\_batch}(#1)\xspace}
\newcommand{\listspanningforest}[1]{\textsc{list\_spanning\_forest}(#1)\xspace}
\newcommand{\cleanup}{\textsc{cleanup}()\xspace}

\DeclareMathOperator\supp{supp}

% proof stuff
\newcommand{\sketchvec}{x}
\newcommand{\boundaryidx}{\beta}
\newcommand{\buck}{b}



\newcommand{\etal}{\text{et al}.\xspace}

\date{}

\newcommand{\newtext}[1]{\textcolor{red}{#1}}

% Uncomment to enable comments
%\newcommand{\namedcomment}[3]{{\sf \scriptsize \color{#2} #1: #3}}
\newcommand{\namedcomment}[3]{{\sf \color{#2} #1: #3}}
% \renewcommand{\namedcomment}[3]{}

\newcommand{\mab}[1]{\namedcomment{mab}{red}{#1}}
\newcommand{\mfc}[1]{\namedcomment{mfc}{purple}{#1}}
\newcommand{\david}[1]{\namedcomment{david}{red}{#1}}
\newcommand{\ahmed}[1]{\namedcomment{ahmed}{blue}{#1}}
\newcommand{\victor}[1]{\namedcomment{victor}{green}{#1}}
\newcommand{\evan}[1]{\namedcomment{evan}{orange}{#1}}
\newcommand{\abi}[1]{\namedcomment{abi}{cyan}{#1}}
\newcommand{\gil}[1]{\namedcomment{gil}{green}{#1}}
\newcommand{\daniel}[1]{\namedcomment{daniel}{cyan}{#1}}


%%%%%%%%%%%%%%%%%%%%
%% MAB: setting up a new way to do comments in margin
%%%%%%%%%%%%%%%
\renewcommand{\mab}[1]{\todo[size=\tiny,color=green!40]{MAB: #1}}
\renewcommand{\mfc}[1]{\todo[size=\tiny,color=green!40]{MFC: #1}}
\renewcommand{\david}[1]{\todo[size=\tiny]{David: #1}}
\renewcommand{\ahmed}[1]{\todo[size=\tiny,color=yellow]{Ahmed: #1}}
\renewcommand{\victor}[1]{\todo[size=\tiny,color=yellow]{Victor: #1}}
\renewcommand{\evan}[1]{\todo[size=\tiny,color=red!40]{Evan: #1}}
\renewcommand{\daniel}[1]{\todo[size=\tiny,color=cyan!40]{Dan: #1}}
\renewcommand{\gil}[1]{\todo[size=\tiny,color=orange!40]{Gil: #1}}
\renewcommand{\abi}[1]{\todo[size=\tiny,color=red!40]{Abi: #1}}
\newcommand{\fixme}[1]{\todo[size=\tiny]{#1}}

\newcommand{\inline}[1]{\todo[inline,color=yellow,size=\tiny]{#1}}

%%%%%%%%%%%%%%%%%%%%%%%%%%%%%%%
%% MAB: iffalse or un-iffalase the following to add or remove our comments
%%%%%%%%%%%%%%%%%%%
\iftrue
\else
%%%%%%%%%%%%%%%%%%%

\renewcommand{\mab}[1]{}
\renewcommand{\mfc}[1]{}
\renewcommand{\david}[1]{}
\renewcommand{\ahmed}[1]{}
\renewcommand{\victor}[1]{}
\renewcommand{\evan}[1]{}
\renewcommand{\abi}[1]{}
\renewcommand{\dan}[1]{}
\renewcommand{\gil}[1]{}
\renewcommand{\fixme}[1]{}
\renewcommand{\inline}[1]{}

%%%%%%%%%%%%%%%%%%%
\fi 
%%%%%%%%%%%%%%%%%%%

% \begin{animateinline}[autoplay,loop]{2}%
%%   \randomcolor{randomcolora}
%%   \randomcolor{randomcolorb}
%%   \randomcolor{randomcolorc}

%%   \noindent\fadingtext{left color=randomcolora,right color=randomcolorb,middle color=randomcolorc!80!black}
%%              {\sf \scriptsize \sloppy \parbox{6.5in}{Rob:  #1}}
%%              %\newframe
%%              %\end{animateinline}
%% }



\newcommand\hcancel[2][black]{\setbox0=\hbox{$#2$}%
\rlap{\raisebox{.45\ht0}{\textcolor{#1}{\rule{\wd0}{1pt}}}}#2} 


\newcommand{\varK}{24\xspace}
\renewcommand{\epsilon}{\varepsilon}
\newcommand{\bet}{B$^{\varepsilon}$-tree\xspace}
\newcommand{\bets}{B$^{\varepsilon}$-trees\xspace}
\newcommand{\pf}{popcorn filter\xspace}

%% References
\newcommand{\appref}[1]         {Appendix~\ref{app:#1}}
\newcommand{\applabel}[1]    {\label{app:#1}}

\newcommand{\chapref}[1]        {Chapter~\ref{chap:#1}}
\newcommand{\secref}[1]         {Section~\ref{sec:#1}}
\newcommand{\seclabel}[1]    {\label{sec:#1}}
\newcommand{\subsecref}[1]      {Subsection~\ref{subsec:#1}}
\newcommand{\subseclabel}[1]    {\label{subsec:#1}}
\newcommand{\secreftwo}[2]      {Sections \ref{sec:#1} and~\ref{sec:#2}}
\newcommand{\secrefthree}[3]    {Sections \ref{sec:#1}, \ref{sec:#2}, and \ref{sec:#3}}
\newcommand{\secreffour}[4]     {Sections \ref{sec:#1}, \ref{sec:#2}, \ref{sec:#3}, and~\ref{sec:#4}}
\newcommand{\quanlabel}[1] {\label{quan:#1}}
\newcommand{\quanref}[1]  {Quantity~\ref{quan:#1}}
\newcommand{\figlabel}[1]   {\label{fig:#1}}
\newcommand{\figref}[1]         {Figure~\ref{fig:#1}}
\newcommand{\figreftwo}[2]      {Figures \ref{fig:#1} and~\ref{fig:#2}}
\newcommand{\tabref}[1]         {Table~\ref{tab:#1}}
\newcommand{\tablabel}[1]   {\label{tab:#1}}
\newcommand{\stref}[1]          {Step~\ref{st:#1}}
\newcommand{\thmlabel}[1]   {\label{thm:#1}}
\newcommand{\thmref}[1]         {Theorem~\ref{thm:#1}}
\newcommand{\thmabbrevref}[1]         {Thm.~\ref{thm:#1}}
\newcommand{\claimlabel}[1]         {\label{claim:#1}}
\newcommand{\claimref}[1]         {Claim~\ref{claim:#1}}
\newcommand{\thmreftwo}[2]      {Theorems \ref{thm:#1} and~\ref{thm:#2}}
\newcommand{\lemlabel}[1]   {\label{lem:#1}}
\newcommand{\lemref}[1]         {Lemma~\ref{lem:#1}}
\newcommand{\algolabel}[1]   {\label{alg:#1}}
\newcommand{\algoref}[1]         {Algorithm~\ref{alg:#1}}
\newcommand{\lemreftwo}[2]      {Lemmas \ref{lem:#1} and~\ref{lem:#2}}
\newcommand{\lemrefthree}[3]    {Lemmas \ref{lem:#1}, \ref{lem:#2}, and~\ref{lem:#3}}
\newcommand{\corlabel}[1]   {\label{cor:#1}}
\newcommand{\corref}[1]         {Corollary~\ref{cor:#1}}
\newcommand{\nonlabel}[1]    {\label{blank:#1}}
\newcommand{\nonref}[1]          {~(\ref{blank:#1})}
\newcommand{\eqlabel}[1]    {\label{eq:#1}}
\newcommand{\eqreff}[1]          {(\ref{eq:#1})}
\renewcommand{\eqref}[1]          {Eq.~\ref{eq:#1}}
\newcommand{\eqreftwo}[2]       {(\ref{eq:#1}) and~(\ref{eq:#2})}
\newcommand{\ineqlabel}[1]    {\label{ineq:#1}}
\newcommand{\ineqref}[1]        {Inequality~(\ref{ineq:#1})}
\newcommand{\ineqreftwo}[2]     {Inequalities (\ref{ineq:#1}) and~(\ref{ineq:#2})}
\newcommand{\invref}[1]         {Invariant~\ref{inv:#1}}
\newcommand{\deflabel}[1]    {\label{def:#1}}
\newcommand{\defref}[1]         {Definition~\ref{def:#1}}
\newcommand{\propref}[1]        {Property~\ref{prop:#1}}
\newcommand{\propreftwo}[2]     {Properties~\ref{prop:#1} and~\ref{prop:#2}}
\newcommand{\proplabel}[1]        {\label{prop:#1}}

\newcommand{\caseref}[1]        {Case~\ref{case:#1}}
\newcommand{\casereftwo}[2]     {Cases \ref{case:#1} and~\ref{case:#2}}
\newcommand{\lilabel}[1]        {\label{li:#1}}
\newcommand{\liref}[1]          {line~\ref{li:#1}}
\newcommand{\Liref}[1]          {Line~\ref{li:#1}}
\newcommand{\lirefs}[2]         {lines \ref{li:#1}--\ref{li:#2}}
\newcommand{\Lirefs}[2]         {Lines \ref{li:#1}--\ref{li:#2}}
\newcommand{\lireftwo}[2]       {lines \ref{li:#1} and~\ref{li:#2}}
\newcommand{\lirefthree}[3]     {lines \ref{li:#1}, \ref{li:#2}, and~\ref{li:#3}}
\newcommand{\exref}[1]          {Exercise~\ref{ex:#1}}
\newcommand{\princref}[1]       {Principle~\ref{prop:#1}}

\newcommand{\obslabel}[1]   {\label{obs:#1}}
\newcommand{\obsref}[1]         {Observation~\ref{obs:#1}}


\newcommand{\resultref}[1]         {Result~\ref{result:#1}}
\newcommand{\resultlabel}[1]   {\label{result:#1}}
\newcommand{\resultreftwo}[2]      {Results~\ref{result:#1} and~\ref{result:#2}}
\newcommand{\resultrefthree}[3]    {Results~\ref{result:#1}, \ref{result:#2}, and~\ref{result:#3}}
\newcommand{\resultrefthrough}[2]      {Results~\ref{result:#1}-\ref{result:#2}}

\newcolumntype{P}[1]{>{\centering\arraybackslash}p{#1}}



\definecolor{bg}{rgb}{0.95,0.95,0.95}
%%% Local Variables:
%%% mode: latex
%%% TeX-master: "main.tex"
%%% End:


\newcommand{\cameospacethm}{Using $3$-wise independent hash functions, \sketchname requires $8\log_{3}(1/\delta) (\log n + 5)$ bytes of space to return a nonzero element of a length $n < 2^{64}$ input vector w/p at least $1-\delta$.}

\newcommand{\restatecameospacethm}{\vspace{\baselineskip} Recall \textsc{Theorem~\ref{thm:cameo_cols}} \emph{\cameospacethm}}

